
\chapter{Capo II}
%_______

\lettrine{T}{re mesi prima,} io era andato a Torino, ed avea riveduto, dopo parecchi
anni di separazione, i miei cari genitori, uno de' fratelli e le due
sorelle. Tutta la nostra famiglia s'era sempre tanto amata! Niun figliuolo
era stato più di me colmato di benefizi dal padre e dalla madre! Oh come al
rivedere i venerati vecchi io m'era commosso, trovandoli notabilmente più
aggravati dall'età che non m'immaginava! Quanto avrei allora voluto non
abbandonarli più, consacrarmi a sollevare colle mie cure la loro vecchiaia!
Quanto mi dolse, ne' brevi giorni ch'io stetti a Torino, di aver parecchi
doveri che mi portavano fuori del tetto paterno, e di dare così poca parte
del mio tempo agli amati congiunti! La povera madre diceva con melanconica
amarezza: «Ah, il nostro Silvio non è venuto a Torino per veder noi!». Il
mattino che ripartii per Milano, la separazione fu dolorosissima. Il padre
entrò in carrozza con me, e m'accompagnò per un miglio; poi tornò indietro
soletto. Io mi voltava a guardarlo, e piangeva, e baciava un anello che la
madre m'avea dato, e mai non mi sentii così angosciato di allontanarmi
da' parenti. Non credulo a' presentimenti, io stupiva di non poter vincere il
mio dolore, ed era sforzato a dire con ispavento: «D'onde questa mia
straordinaria inquietudine?». Pareami pur di prevedere qualche grande
sventura.

Ora, nel carcere, mi risovvenivano quello spavento, quell'angoscia; mi
risovvenivano tutte le parole udite, tre mesi innanzi, da' genitori. Quel
lamento della madre: «Ah! Il nostro Silvio non è venuto a Torino per veder
noi!» mi ripiombava sul cuore. Io mi rimproverava di non essermi mostrato
loro mille volte più tenero.~---~Li amo cotanto, e ciò dissi loro così
debolmente! Non dovea mai più vederli, e mi saziai così poco de' loro cari
volti! E fui così avaro delle testimonianze dell'amor mio!~---~Questi
pensieri mi straziavano l'anima.

Chiusi la finestra, passeggiai un'ora, credendo di non aver requie tutta la
notte. Mi posi a letto, e la stanchezza m'addormentò.
