
\chapter{Capo I}

%_______

\lettrine{I}{l venerdì} 13 ottobre 1820 fui arrestato a Milano, e condotto a Santa
Margherita. Erano le tre pomeridiane. Mi si fece un lungo interrogatorio
per tutto quel giorno e per altri ancora. Ma di ciò non dirò nulla. Simile
ad un amante maltrattato dalla sua bella, e dignitosamente risoluto di
tenerle broncio, lascio la politica ov'ella sta, e parlo d'altro.

Alle nove della sera di quel povero venerdì, l'attuario mi consegnò al
custode, e questi, condottomi nella stanza a me destinata, si fece da me
rimettere con gentile invito, per restituirmeli a tempo debito, orologio,
denaro, e ogni altra cosa ch'io avessi in tasca, e m'augurò rispettosamente
la buona notte.

---~Fermatevi, caro voi, gli dissi; oggi non ho pranzato; fatemi portare
qualche cosa.

---~Subito, la locanda è qui vicina; e sentirà, signore, che buon vino!

---~Vino, non ne bevo.

A questa risposta, il signor Angiolino mi guardò spaventato, e sperando
ch'io scherzassi: I custodi di carceri che tengono bettola, inorridiscono
d'un prigioniero astemio.

---~Non ne bevo, davvero.

---~M'incresce per lei; patirà al doppio la solitudine\dots{}

E vedendo ch'io non mutava proposito, uscì; ed in meno di mezz'ora ebbi il
pranzo. Mangiai pochi bocconi, tracannai un bicchier d'acqua, e fui
lasciato solo.

La stanza era a pian terreno, e metteva sul cortile. Carceri di qua,
carceri di là, carceri di sopra, carceri dirimpetto. M'appoggiai alla
finestra, e stetti qualche tempo ad ascoltare l'andare e venire de'
carcerieri, ed il frenetico canto di parecchi de' rinchiusi.

Pensava:~---~Un secolo fa, questo era un monastero: avrebbero mai le sante e
penitenti vergini che lo abitavano, immaginato che le loro celle
suonerebbero oggi, non più di femminei gemiti e d'inni divoti, ma di
bestemmie e di canzoni invereconde, e che conterrebbero uomini d'ogni
fatta, e per lo più destinati agli ergastoli o alle forche? E fra un
secolo, chi respirerà in queste celle? Oh fugacità del tempo! oh mobilità
perpetua delle cose! Può chi vi considera affliggersi, se fortune cessò di
sorridergli, se vien sepolto in prigione, se gli si minaccia il patibolo?
Jeri, io era uno de' più felici mortali del mondo: oggi non ho più alcuna
delle dolcezze che confortavano la mia vita; non più libertà, non più
consorzio d'amici, non più speranze! No; il lusingarsi sarebbe follia. Di
qui non uscirò se non per essere gettato ne' più orribili covili, o
consegnato al carnefice! Ebbene, il giorno dopo la mia morte, sarà come
s'io fossi spirato in un palazzo, e portato alla sepoltura co' più grandi
onori.~---

Così il riflettere alla fugacità del tempo m'invigoriva l'animo. Ma mi
ricorsero alla mente il padre, la madre, due fratelli, due sorelle,
un'altra famiglia ch'io amava quasi fosse la mia; ed i ragionamenti
filosofici nulla più valsero. M'intenerii, e piansi come un fanciullo.
